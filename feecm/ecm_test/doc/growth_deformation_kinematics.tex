%% LyX 2.3.4.2 created this file.  For more info, see http://www.lyx.org/.
%% Do not edit unless you really know what you are doing.
\documentclass[english]{article}
\usepackage[T1]{fontenc}
\usepackage[latin9]{inputenc}
\usepackage{refstyle}
\usepackage{amsmath}

\makeatletter

%%%%%%%%%%%%%%%%%%%%%%%%%%%%%% LyX specific LaTeX commands.

\AtBeginDocument{\providecommand\eqref[1]{\ref{eq:#1}}}
\AtBeginDocument{\providecommand\subsecref[1]{\ref{subsec:#1}}}
\RS@ifundefined{subsecref}
  {\newref{subsec}{name = \RSsectxt}}
  {}
\RS@ifundefined{thmref}
  {\def\RSthmtxt{theorem~}\newref{thm}{name = \RSthmtxt}}
  {}
\RS@ifundefined{lemref}
  {\def\RSlemtxt{lemma~}\newref{lem}{name = \RSlemtxt}}
  {}


\makeatother

\usepackage{babel}
\begin{document}
\title{Kinematics of Growth Deformation}

\maketitle
Consider a macroscopically-homogeneous body \textbf{B} with the region
of space it occupies in a fixed reference configuration, and denote
by \textbf{X} an arbitrary material point of \textbf{B}. A motion
of \textbf{B} is then a smooth one-to-one mapping $\mathbf{x}=\mathbf{\mathbf{\chi}}(\mathbf{X},t)$
with\emph{ deformation gradient, velocity, and velocity gradient}
given by
\begin{equation}
\mathbf{F}=\nabla\mathbf{\chi},\mathbf{v}=\mathbf{\dot{{\mathbf{\chi}}}},\mathbf{L}=\mathrm{grad}\mathbf{v}=\mathbf{\dot{F}F^{-1}}\label{eq:def_grad_defn}
\end{equation}

Assume that $J=\mathrm{det}\mathbf{F}>0$, where $J=\frac{dV}{dV_{0}}$.
Following modern developments in large-deformation plasticity and
growth we regard the growth of an interphase layer on the multiplicative
decomposition of the deformation gradient, 

\begin{equation}
\mathbf{F=F^{e}F^{p}F^{g}}\label{eq:def_grad_decomp}
\end{equation}

Here
\begin{enumerate}
\item $\mathbf{F^{g}(X},t)$ represents the local distortion of the material
neighborhood of $\mathbf{X}$ due to growth by deposition of Li metal;
\item $\mathbf{F^{p}(X},t)$ represents the constant volume plastic (viscoplastic)
deformation of the deposited Li;
\item $\mathbf{F^{e}(X},t)$ represents the elastic distortion that occurs
due to the subsequent stretching and rotation. 
\end{enumerate}
$J$ now follows the multiplicative decomposition of the deformation
gradient according to 

\begin{equation}
J=J^{e}J^{p}J^{g}\label{eq:J_decomp}
\end{equation}

where $J^{e}=\mathrm{det}\mathbf{F}^{e}>0$, $J^{p}=\mathrm{det}\mathbf{F}^{p}=1$
(constant volume deformation), and $J^{g}=\mathrm{det}\mathbf{F}^{g}>0$
. We further define the elastic, plastic and growth stretch and spin
tensors through 

\begin{equation}
\begin{array}{cc}
\mathbf{D}^{e}=\mathrm{symm}\mathbf{L}^{e}, & \mathbf{W}^{e}=\mathrm{skew}\mathbf{L}^{e}\\
\mathbf{D}^{p}=\mathrm{symm}\mathbf{L}^{p}, & \mathbf{W}^{p}=\mathrm{skew}\mathbf{L}^{p}\\
\mathbf{D}^{g}=\mathrm{symm}\mathbf{L}^{g}, & \mathbf{W}^{g}=\mathrm{skew}\mathbf{L}^{g}
\end{array}\label{eq:stretch_spin_tensors_defn}
\end{equation}

where $\mathbf{L=L}^{e}+\mathbf{F}^{e}\mathbf{L}^{p}\mathbf{F}^{e-1}+(\mathbf{F}^{e}\mathbf{F}^{p})\mathbf{L}^{g}(\mathbf{F}^{e}\mathbf{F}^{p})^{-1}$and
$\mathbf{L}^{e}=\mathbf{F}^{e}\mathbf{F}^{e-1},$$\mathbf{L}^{p}=\mathbf{F}^{p}\mathbf{F}^{p-1}$,
$\mathbf{L}^{g}=\mathbf{F}^{g}\mathbf{F}^{g-1}$. We further assume
that the plastic spin and growth spin tensors are identically zero. 

\section{Growth Deformation }

We introduce a set of orthonormal unit vectors$\{\mathbf{m}_{Ri}\vert i=1,2,3\}$
in the reference body and consider a particular class of growth tensors
$\mathbf{F}^{g}$ defined by 

\begin{equation}
\mathbf{F}^{g}=\sum_{i=1}^{3}\lambda_{i}^{g}(\mathbf{m}_{Ri}\otimes\mathbf{m}_{Ri})\label{eq:Fg_defn}
\end{equation}

where $\{\lambda_{i}^{g}>0\vert i=1,2,3\}$ are the corresponding
growth stretches which satisfy 

\begin{equation}
\prod_{i=1}^{3}\lambda_{i}^{g}=J^{g}\label{eq:Jg_defn}
\end{equation}

Since the growth is assumed to have no spin we can define from \ref{eq:Fg_defn} 

\begin{equation}
\begin{aligned}\dot{\mathbf{F}^{g}=}\sum_{i=1}^{3}\lambda_{i}^{g}(\mathbf{m}_{Ri}\otimes\mathbf{m}_{Ri}),\mathrm{and} & \mathbf{F}^{g-1}=\sum_{i=1}^{3}\lambda_{i}^{g-1}(\mathbf{m}_{Ri}\otimes\mathbf{m}_{Ri})\end{aligned}
\label{eq:growth_def_grad_rate}
\end{equation}

From \ref{eq:growth_def_grad_rate} we can now define the growth stretch
rate as 

\begin{equation}
\begin{split}\begin{split}\mathbf{D}^{g}=\sum_{i=1}^{3}\dot{\epsilon_{i}^{g}} & (\mathbf{m}_{Ri}\otimes\mathbf{m}_{Ri})\end{split}
\mathrm{and} & \mathbf{W}^{g}=0\end{split}
\label{eq:growth_stretch_rate_defn}
\end{equation}

where $\dot{\epsilon}_{i}^{g}=\dot{\lambda_{i}^{g}\lambda_{i}^{g}}$
and $\mathrm{tr}\mathbf{D}^{g}=\dot{\epsilon^{g}}=\sum_{i=1}^{3}\dot{\epsilon_{i}^{g}}=\dot{J^{g}J^{g-1}}$.
Here $\dot{\epsilon^{g}}$is a volumetric growth strain rate and volumetric
growth strain is then given by 

\begin{equation}
\epsilon_{g}=\ln J^{g}\label{eq:vol_growth_strain}
\end{equation}

If $\dot{\epsilon^{g}}>0$ material growth or Li is deposited and
$\dot{\epsilon^{g}<0}$ materials shrinks or Li is stripped. 

To allow for differential rates along the preferred directions $\{\mathbf{m}_{Ri}\vert i=1,2,3\}$
we introduce a triplet of parameters$\{\alpha_{i}\in[0,1]\vert i=1,2,3\}$
and assume that the growth strain rate in each direction is a fraction
of the volumetric growth strain rate 

\begin{equation}
\dot{\epsilon}_{i}^{g}=\alpha_{i}\dot{\epsilon}_{i}^{g}\label{eq:growth_weights_defn}
\end{equation}
The growth weigths $\alpha_{i}$ must satisfy the constraint 

\begin{equation}
\sum_{i=1}^{3}\alpha_{i}=1\label{eq:growth_weight_constraint}
\end{equation}

Now using the above equations we note that 

\begin{equation}
\dot{\lambda}_{i}^{g}\lambda_{i}^{g-1}=\alpha_{i}\dot{J}^{g}J^{g-1}\label{eq:growth_weight_Jg}
\end{equation}

which can now be integrated to give 

\begin{equation}
\lambda_{i}^{g}=\left(J^{g}\right)^{\alpha_{i}}\label{eq:individual_growth_strethc}
\end{equation}

Finally, all the above arguments can now be condensed to the rate
form for the evolution of $\mathbf{F}^{g}$ as 

\begin{equation}
\dot{\mathbf{F}}^{g}=\mathbf{D}^{g}\mathbf{F}^{g}\label{eq:Fg_evolution}
\end{equation}

with 

\begin{equation}
\mathbf{D}^{g}=\dot{\epsilon}^{g}\mathbf{S}^{g}\label{eq:Dg_eqn}
\end{equation}

where $\mathbf{S}^{g}$ is the principal growth tensor and defined
as 

\begin{equation}
\mathbf{S}^{g}=\sum_{i=1}^{3}\alpha_{i}(\mathbf{m}_{Ri}\otimes\mathbf{m}_{Ri})\label{eq:Sg_reference}
\end{equation}


\subsection{Non-fixed reference vector}

We can now extend this analysis to a non-fixed reference vector defined
in the intermediate configuration of $\mathbf{F}^{g}$. We replace
$\mathbf{m}_{Ri}$ the reference vector by $\mathbf{m}_{i}^{g}$ which
are orthonormal unit vectors in the intermediate space of $\boldsymbol{F}_{g}$.
Thus the principal growth tensor in \ref{eq:Sg_reference} is now
defined in the intermediate space as 

\begin{equation}
\mathbf{S}^{g}=\sum_{i=1}^{3}\alpha_{i}(\mathbf{m}_{i}^{g}\otimes\mathbf{m}_{i}^{g})\label{eq:principal_growth_tensor_interme}
\end{equation}

\begin{description}
\item [{Warning:}] With a non-fixed reference vector the direct formulation
for $\mathrm{\boldsymbol{F}}_{g}$ \eqref{Fg_defn} cannot be used
any longer and we have to rely on the incremental formulation. 
\end{description}

\subsubsection*{Axial (Fiber) growth\label{subsec:Axial-(Fiber)-growth}}

This is a specialization of the equation for growth along a particular
axis. Instead of defining all three principal directions, we define
a single principal (fiber) direction $\mathrm{\boldsymbol{m}}_{1}$
along which growth occurs. For this form of growth, 

\[
\begin{aligned}\alpha_{1}=1, & \alpha_{2}=\alpha_{3}=0,\end{aligned}
\mathrm{so\;that}
\]

\begin{equation}
\mathbf{S}^{g}=\left(\mathrm{\boldsymbol{m}}_{1}\otimes\mathrm{\boldsymbol{m}}_{1}\right)\label{eq:axial_Sg_fiber_direction}
\end{equation}


\subsubsection*{Areal growth (Growth perpendicular to a single principal direction)}

A similar specilization to the previous section \subsecref{Axial-(Fiber)-growth}
can be applied to growth perpendicular to a single principal direction
$\mathrm{\boldsymbol{m}}_{1}$. For this form of growth, 

\[
\begin{aligned}\alpha_{1}=0, & \alpha_{2}=\alpha_{3}=\frac{1}{2}\end{aligned}
,\mathrm{so\;that}
\]

\begin{equation}
\mathrm{\boldsymbol{S}}_{g}=(\boldsymbol{1}-\mathrm{\boldsymbol{m}_{1}}\otimes\mathrm{\boldsymbol{m}_{1}})\label{eq:Sg_area_growth}
\end{equation}

In order to solve for the evolution of $\mathrm{\boldsymbol{F}}_{g}$
\eqref{Fg_evolution} we assume that at the start of deformation the
initial condition for $\mathrm{\boldsymbol{F}}_{g}(\mathrm{\boldsymbol{X}},t=0)=\mathrm{\boldsymbol{1}}$.
Therefore, 

\begin{equation}
\mathrm{\boldsymbol{F}_{g}(t+\Delta t})=\exp(\mathrm{\boldsymbol{D}_{g}\Delta t})\mathrm{\boldsymbol{F}_{g}(t)}\label{eq:F_g_actual_equation}
\end{equation}


\subsubsection*{Constitutive equation for swelling }

The swelling volume change $J^{g}$ is related to the Li concentration
in the interphase layer through a linear constitutive equation 

\begin{equation}
J^{g}=1+\Omega(c_{R}-c_{R0})\label{eq:Jg_constitutive}
\end{equation}

where $c_{R}$is the concentration, $c_{R0}$ is a reference concentration,
with $\Omega>0$ representing the \emph{molar volume of Li. }From
\eqref{vol_growth_strain} and \eqref{Jg_constitutive} we now have 

\begin{equation}
\dot{J}^{g}J^{g}=\Theta(c_{R})\dot{c}_{R}\label{eq:Jg_evolution_const}
\end{equation}

where

\begin{equation}
\Theta(c_{R})=\frac{\Omega}{1+\Omega c_{R}}\label{eq:theta_const}
\end{equation}

we obtain the volumetric growth strain rate as 

\begin{equation}
\dot{\epsilon}^{g}=\Theta(c_{R})\dot{c}_{R}\label{eq:vol_sgrowth_train_rate_const}
\end{equation}

From \eqref{vol_sgrowth_train_rate_const} and \eqref{theta_const}
the growth stretch rate of \eqref{Dg_eqn} can now be written as 

\begin{equation}
\mathrm{\boldsymbol{D}}^{g}=\Theta(c_{R})\dot{c}_{R}\mathrm{\boldsymbol{S}_{g}}\label{eq:Dg_const_defn}
\end{equation}


\section{Li Flux}

The Li flux in the reference fram is presumed to obey the constitutive
equation 

\begin{equation}
\mathrm{\boldsymbol{j}_{R}}=-\left[\sum_{i=1}^{3}m_{i}(c_{R})\mathrm{\boldsymbol{m}}_{Ri}\otimes\mathrm{\boldsymbol{m}}_{Ri}\right]\nabla\mu\label{eq:reference_Li_flux}
\end{equation}

where $m_{i}$ is a scalar mobility of Li in the $\mathrm{\boldsymbol{m}}_{R}$-direction.
The spatial flux is related to the reference flux by 

\begin{equation}
\mathrm{\boldsymbol{j}}_{R}=J\mathrm{\boldsymbol{F}}^{-1}\mathrm{\boldsymbol{j}}\label{eq:ref_flux_spatial_flux}
\end{equation}

and 

\begin{equation}
\nabla\mu=\mathrm{\boldsymbol{F}}^{T}\mathrm{grad}\mu\label{eq:gradient_chem_pot_ref_spat}
\end{equation}

From \eqref{reference_Li_flux}, \eqref{ref_flux_spatial_flux} and
\eqref{gradient_chem_pot_ref_spat} we can now write the spatial flux
as 

\begin{equation}
\mathrm{\boldsymbol{j}}=-\left[J^{-1}\sum_{i=1}^{3}m_{i}(c_{R})\mathrm{\boldsymbol{m}}_{i}\otimes\mathrm{\boldsymbol{m}}_{i}\right]\mathrm{grad}\mu\label{eq:spatial_flux}
\end{equation}

with $\mathrm{\boldsymbol{m}}_{i}=\mathrm{\boldsymbol{F}}\mathrm{\boldsymbol{m}}_{Ri}$
and $\mu$ is the chemical potential. 

The chemical potential can take many forms 
\begin{enumerate}
\item $\mu=\mu_{0}+R\vartheta\log\left(\frac{\gamma c_{R}}{c_{R0}}\right)+\Omega(\frac{1}{2}\mathrm{\boldsymbol{E}}^{e}:\mathcal{C}\mathrm{\boldsymbol{E}}^{e})-\omega$,
where $\mathrm{\boldsymbol{E}^{e}}$ is the logarithmic elastic strain,
$\mathcal{C}$ is the isotropic elasticity tensor, $\vartheta$ is
the temperature, $\gamma$ is the activity coefficient, $R$ is the
universal gas constant, and $\omega$ is the growth chemical potential 
\item $\mu=R\vartheta\ln\left(\frac{\bar{c}}{1-\bar{c}}\right)+\Omega(\frac{1}{2}\mathrm{\boldsymbol{E}}^{e}:\mathcal{C}\mathrm{\boldsymbol{E}}^{e})-\omega$
\end{enumerate}
For the model described above 

\begin{equation}
\omega=\Omega\mathrm{\boldsymbol{M}^{g}:\mathrm{\boldsymbol{S}}^{g}}\label{eq:growth_chemical_potential}
\end{equation}

where 

\begin{equation}
\mathrm{\boldsymbol{M}}^{g}=\mathrm{\boldsymbol{F}}^{pT}\mathrm{\boldsymbol{M}}^{e}\mathrm{\boldsymbol{F}}^{p-T}\label{eq:growth_stress_eqn}
\end{equation}

and 

\begin{equation}
\mathrm{\boldsymbol{M}}^{e}=J^{e}\mathrm{\boldsymbol{R}}^{eT}\mathrm{\boldsymbol{\sigma}}\mathrm{\boldsymbol{R}}^{e}\label{eq:Mandel_cauchy_stress_rel}
\end{equation}

is the Mandel stress. The description of these quantities and the
thermodynamic derivation of these quantities is beyond the scope of
this document. 

In Li dominated problems under small stress limits, the growth stress
and the elastic stresses are expected to be small, so both the growth
and the elastic terms can be ignored for the chemical potential. 


\end{document}
