\documentclass[12pt,a4paper]{article}
\usepackage{amsmath}
\usepackage{amsfonts}
\usepackage{amssymb}
\usepackage{graphicx}
\usepackage{geometry}
\usepackage{fancyhdr}
\usepackage{hyperref}

\geometry{margin=1in}
\pagestyle{fancy}
\fancyhf{}
\rhead{Case 1: Formation of EDL}
\lhead{TECM\_TEST Framework}
\cfoot{\thepage}

\title{Case 1: Formation of Electric Double Layer (EDL) \\
in Binary Ion System}
\author{TECM\_TEST Documentation}
\date{\today}

\begin{document}

\maketitle

\section{Introduction}

This document outlines the theoretical framework and implementation strategy for modeling the formation of an Electric Double Layer (EDL) in a binary ion system using the TECM\_TEST framework. The system consists of two ionic species: cation "c" and anion "a".

\section{Physical System Description}

\subsection{System Definition}
The binary electrolyte system contains:
\begin{itemize}
    \item \textbf{Cation (c)}: Positively charged ion with valence $z_c = +1$
    \item \textbf{Anion (a)}: Negatively charged ion with valence $z_a = -1$
    \item \textbf{Solvent}: Dielectric medium with permittivity $\varepsilon$
\end{itemize}

\subsection{EDL Formation Mechanism}
When a charged surface is introduced into the electrolyte:
\begin{enumerate}
    \item Surface charge creates local electric field
    \item Counter-ions accumulate near the surface
    \item Co-ions are repelled from the surface
    \item Concentration gradients develop
    \item Steady-state EDL structure forms
\end{enumerate}

\section{Governing Equations}

The EDL formation in the binary ion system is governed by three coupled partial differential equations:

\subsection{Species Transport - Nernst-Planck Equations}

For cation transport:
\begin{equation}
    \frac{\partial c_c}{\partial t} + \nabla \cdot \mathbf{J}_c = 0
\end{equation}

For anion transport:
\begin{equation}
    \frac{\partial c_a}{\partial t} + \nabla \cdot \mathbf{J}_a = 0
\end{equation}

Where the ionic flux is given by:
\begin{equation}
    \mathbf{J}_i = -D_i \nabla c_i - \frac{z_i F D_i}{RT} c_i \nabla \Phi
\end{equation}

The flux consists of:
\begin{itemize}
    \item \textbf{Diffusion term}: $-D_i \nabla c_i$
    \item \textbf{Migration term}: $-\frac{z_i F D_i}{RT} c_i \nabla \Phi$
\end{itemize}

\subsection{Electrostatic Potential - Poisson Equation}

The electric potential is governed by Poisson's equation:
\begin{equation}
    -\nabla \cdot (\varepsilon \nabla \Phi) = F(z_c c_c + z_a c_a)
\end{equation}

This couples the electric field to the local charge density.

\subsection{Electroneutrality Condition}

In the bulk electrolyte (initial condition):
\begin{equation}
    z_c c_c + z_a c_a = 0
\end{equation}

For the binary system: $c_c^0 = c_a^0 = c_0$ (bulk concentration)

\section{Analytical Solution of GCS Theory under Low Voltage}

\subsection{EDL Structure}
The electric double layer consists of two distinct regions:
\begin{itemize}
    \item \textbf{Stern Layer (Compact Layer)}: Thickness $x_S$, permittivity $\varepsilon_{r,S}$
    \item \textbf{Diffuse Layer}: Characteristic thickness $x_D$ (Debye length), permittivity $\varepsilon_{r,D}$
\end{itemize}

\subsection{Low Potential Analytical Solution}
For low applied potential ($\phi \ll RT/F \approx 25$ mV), the Poisson-Boltzmann equation can be linearized.

\subsubsection{Gouy-Chapman Equation for Diffuse Layer}
Starting with Poisson's equation:
\begin{equation}
    -\varepsilon_{r,D}\varepsilon_0 \frac{d^2\phi}{dx^2} = F(c_+ - c_-)
\end{equation}

For low potential, the Boltzmann distribution linearizes:
\begin{equation}
    c_{\pm} \approx c_0\left(1 \mp \frac{F\phi}{RT}\right)
\end{equation}

Therefore:
\begin{equation}
    c_+ - c_- \approx -2c_0\frac{F\phi}{RT}
\end{equation}

Substituting into Poisson's equation:
\begin{equation}
    -\varepsilon_{r,D}\varepsilon_0 \frac{d^2\phi}{dx^2} = -2c_0\frac{F^2\phi}{RT}
\end{equation}

This simplifies to:
\begin{equation}
    \frac{d^2\phi}{dx^2} = \frac{2c_0F^2}{RT\varepsilon_{r,D}\varepsilon_0}\phi = \frac{\phi}{x_D^2}
\end{equation}

where we use the Debye length definition from Equation (1).

The analytical solution is:
\begin{equation}
    \phi(x) = \phi_0 e^{-x/x_D}
\end{equation}

showing exponential potential decay from the electrode surface.

\subsection{Capacitance Derivation}

\subsubsection{Diffuse Layer Capacitance}
The surface charge density at the electrode is:
\begin{equation}
    \sigma = -\varepsilon_{r,D}\varepsilon_0\left.\frac{d\phi}{dx}\right|_{x=0} = \varepsilon_{r,D}\varepsilon_0\frac{\phi_0}{x_D}
\end{equation}

Therefore, the diffuse layer capacitance is:
\begin{equation}
    C_{diffuse} = \frac{\sigma}{\phi_0} = \frac{\varepsilon_{r,D}\varepsilon_0}{x_D}
\end{equation}

\subsubsection{Stern Layer Capacitance}
For a parallel plate capacitor with thickness $x_S$:
\begin{equation}
    C_{Stern} = \frac{\varepsilon_{r,S}\varepsilon_0}{x_S}
\end{equation}

\subsubsection{Total EDL Capacitance}
The total EDL capacitance is the series combination:
\begin{equation}
    \frac{1}{C_{dl}} = \frac{1}{C_{Stern}} + \frac{1}{C_{diffuse}} = \frac{x_S}{\varepsilon_{r,S}\varepsilon_0} + \frac{x_D}{\varepsilon_{r,D}\varepsilon_0}
\end{equation}

This yields the Gouy-Chapman-Stern capacitance formula:
\begin{equation}
    C_{dl} = \left(\frac{x_D}{\varepsilon_{r,D}\varepsilon_0} + \frac{x_S}{\varepsilon_{r,S}\varepsilon_0}\right)^{-1}
\end{equation}

\subsection{Physical Interpretation}
\begin{itemize}
    \item The Stern layer contributes capacitive resistance: $\frac{x_S}{\varepsilon_{r,S}\varepsilon_0}$
    \item The diffuse layer contributes capacitive resistance: $\frac{x_D}{\varepsilon_{r,D}\varepsilon_0}$
    \item Lower permittivity in Stern layer ($\varepsilon_{r,S} = 10$) vs diffuse layer ($\varepsilon_{r,D} = 78.5$)
    \item Series combination because charge must pass through both layers
\end{itemize}

\subsection{Validity of Low Potential Approximation}
This analytical solution is valid when:
\begin{itemize}
    \item Applied potential $\phi \ll RT/F \approx 25$ mV at room temperature
    \item Ion concentrations remain close to bulk values
    \item No specific adsorption or finite ion size effects
    \item Linear response regime
\end{itemize}

At higher potentials, nonlinear effects become important and numerical solutions are required.

\section{Material Properties Required}

\subsection{Transport Properties}
\begin{align}
    D_c &: \text{Cation diffusivity} \quad [\text{m}^2/\text{s}] \\
    D_a &: \text{Anion diffusivity} \quad [\text{m}^2/\text{s}] \\
    z_c &= +1 : \text{Cation valence} \\
    z_a &= -1 : \text{Anion valence}
\end{align}

\subsection{Physical Constants}
\begin{align}
    F &= 96485 \quad [\text{C/mol}] \quad \text{Faraday constant} \\
    R &= 8.314 \quad [\text{J/(mol·K)}] \quad \text{Gas constant} \\
    T &: \text{Temperature} \quad [\text{K}] \\
    \varepsilon &= \varepsilon_r \varepsilon_0 \quad [\text{F/m}] \quad \text{Permittivity}
\end{align}

\section{Current TECM\_TEST Implementation Status}

\subsection{Available Components}
\begin{itemize}
    \item \texttt{PoissonEquation} - Implements Poisson equation with charge density coupling
    \item \texttt{ADNernstPlanckConvection} - Implements migration term in Nernst-Planck
    \item \texttt{ChargeDensity} - Computes multi-species charge density
    \item \texttt{BulkChargeTransport} \& \texttt{Migration} - Electrical conductivity materials
\end{itemize}

\subsection{Missing Components for EDL Modeling}
\begin{enumerate}
    \item \textbf{Diffusion Kernel}: Pure diffusion equation $\frac{\partial c}{\partial t} - D\nabla^2 c = 0$
    \item \textbf{Time Derivative Kernels}: Transient terms $\frac{\partial c}{\partial t}$
    \item \textbf{Species-Specific Properties}: Individual diffusivities $D_c$, $D_a$
    \item \textbf{EDL Boundary Conditions}: Surface charge/potential boundary conditions
    \item \textbf{Example Input File}: Complete setup for binary EDL problem
\end{enumerate}

\section{Implementation Strategy}

\subsection{Phase 1: Core Physics Implementation}
\begin{enumerate}
    \item Create diffusion kernel for species transport
    \item Implement time derivative kernels
    \item Develop species-specific material properties
    \item Set up boundary conditions for charged surfaces
\end{enumerate}

\subsection{Phase 2: Validation and Testing}
\begin{enumerate}
    \item Create benchmark EDL problem
    \item Compare with analytical solutions (Gouy-Chapman theory)
    \item Validate concentration profiles and potential distribution
    \item Test different ionic strengths and surface charges
\end{enumerate}

\subsection{Phase 3: Advanced Features}
\begin{enumerate}
    \item Non-linear EDL effects (finite ion size)
    \item Multiple ion species
    \item Dynamic EDL formation kinetics
    \item Coupling with mechanical deformation
\end{enumerate}


\section{Expected Outcomes}

Upon successful implementation, the TECM\_TEST framework will be capable of:
\begin{itemize}
    \item Modeling EDL formation in binary electrolytes
    \item Predicting ion concentration profiles near charged interfaces
    \item Calculating electric potential distribution in EDL
    \item Studying transient EDL dynamics
    \item Validating against analytical GCS theory for low potentials
    \item Providing foundation for more complex electrochemical systems
\end{itemize}

\section{References}

\begin{enumerate}
    \item Bard, A. J., \& Faulkner, L. R. (2001). \textit{Electrochemical Methods: Fundamentals and Applications}. Wiley.
    \item Newman, J., \& Thomas-Alyea, K. E. (2004). \textit{Electrochemical Systems}. Wiley.
    \item Bazant, M. Z., Thornton, K., \& Ajdari, A. (2004). Diffuse-charge dynamics in electrochemical systems. \textit{Physical Review E}, 70(2), 021506.
\end{enumerate}

\end{document}