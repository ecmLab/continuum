\documentclass[11pt]{article}
\usepackage{amsmath}
\usepackage{amssymb}
\usepackage{geometry}
\geometry{margin=1in}

\title{Analytical Transient Solution for 1D ECRAM System}
\author{TECM-no-EN Framework}
\date{\today}

\begin{document}

\maketitle

\section{Governing System}

\subsection{Time-dependent equations}
\begin{align}
\frac{\partial c_{\text{Li}}}{\partial t} &= D_{\text{Li}} \nabla^2 c_{\text{Li}} + \frac{D_{\text{Li}} F}{RT} \nabla \cdot (c_{\text{Li}} \nabla \Psi) \\
\frac{\partial c_a}{\partial t} &= D_a \nabla^2 c_a - \frac{D_a F}{RT} \nabla \cdot (c_a \nabla \Psi) \\
\nabla^2 \Psi &= -\frac{F}{\varepsilon}(c_{\text{Li}} - c_a)
\end{align}

\subsection{Boundary conditions}
\begin{align}
\Psi(0,t) &= V_{\text{gate}}, \quad \Psi(L,t) = 0 \\
\left.\frac{\partial c_{\text{Li}}}{\partial x}\right|_{x=0,L} &= 0, \quad \left.\frac{\partial c_a}{\partial x}\right|_{x=0,L} = 0
\end{align}

\subsection{Initial conditions}
\begin{align}
c_{\text{Li}}(x,0) &= c_{\text{Li}}^{\text{initial}} \\
c_a(x,0) &= c_a^{\text{initial}}
\end{align}

\section{Analytical Transient Solution}

\subsection{General form (separation of variables)}

\begin{align}
c_{\text{Li}}(x,t) &= c_{\text{Li}}^{\text{ss}}(x) + \sum_{n=1}^{\infty} A_n^{\text{Li}} \varphi_n(x) \exp(-\lambda_n t) \\
c_a(x,t) &= c_a^{\text{ss}}(x) + \sum_{n=1}^{\infty} A_n^a \varphi_n(x) \exp(-\lambda_n t) \\
\Psi(x,t) &= \Psi^{\text{ss}}(x) + \sum_{n=1}^{\infty} B_n \psi_n(x) \exp(-\lambda_n t)
\end{align}

Where:
\begin{itemize}
\item $c_i^{\text{ss}}(x)$: Steady-state Boltzmann distributions
\item $\Psi^{\text{ss}}(x)$: Steady-state potential (nonlinear solution)
\item $\varphi_n(x), \psi_n(x)$: Eigenfunctions of the linearized system
\item $\lambda_n$: Eigenvalues (decay rates)
\item $A_n, B_n$: Coefficients from initial conditions
\end{itemize}

\section{Steady-State Solution}

\subsection{Boltzmann distributions}
\begin{align}
c_{\text{Li}}^{\text{ss}}(x) &= c_{\text{Li}}^0 \exp\left(-\frac{F\Psi^{\text{ss}}(x)}{RT}\right) \\
c_a^{\text{ss}}(x) &= c_a^0 \exp\left(+\frac{F\Psi^{\text{ss}}(x)}{RT}\right)
\end{align}

\subsection{Zero flux conditions}
\begin{align}
J_{\text{Li}} &= -D_{\text{Li}}\frac{dc_{\text{Li}}}{dx} - \frac{D_{\text{Li}} F}{RT} c_{\text{Li}} \frac{d\Psi}{dx} = 0 \\
J_a &= -D_a\frac{dc_a}{dx} + \frac{D_a F}{RT} c_a \frac{d\Psi}{dx} = 0
\end{align}

\section{Eigenvalue Problem}

The decay rates $\lambda_n$ satisfy:
\begin{align}
&\left[D_{\text{Li}} \frac{\partial^2}{\partial x^2} + \frac{D_{\text{Li}} F}{RT}\left(\frac{\partial \Psi^{\text{ss}}}{\partial x} \frac{\partial}{\partial x} + \frac{\partial^2 \Psi^{\text{ss}}}{\partial x^2} \cdot\right)\right]\varphi_n^{\text{Li}} = \lambda_n \varphi_n^{\text{Li}} \\
&\left[D_a \frac{\partial^2}{\partial x^2} - \frac{D_a F}{RT}\left(\frac{\partial \Psi^{\text{ss}}}{\partial x} \frac{\partial}{\partial x} + \frac{\partial^2 \Psi^{\text{ss}}}{\partial x^2} \cdot\right)\right]\varphi_n^a = \lambda_n \varphi_n^a
\end{align}

With the Poisson coupling:
\begin{equation}
\frac{\partial^2 \psi_n}{\partial x^2} = -\frac{F}{\varepsilon}(\varphi_n^{\text{Li}} - \varphi_n^a)
\end{equation}

\section{Current Transient}

\subsection{Total current}
\begin{equation}
I(t) = I_0 \sum_{n=1}^{\infty} C_n \exp(-\lambda_n t)
\end{equation}

Where:
\begin{itemize}
\item $I_0$: Initial current magnitude
\item $C_n$: Modal amplitudes
\item $\lambda_1 < \lambda_2 < \lambda_3 < \ldots$: Ordered eigenvalues
\end{itemize}

\subsection{Current density}
\begin{equation}
j(x,t) = F\left[z_{\text{Li}} J_{\text{Li}}(x,t) + z_a J_a(x,t)\right]
\end{equation}

\section{Time Scale Analysis}

The system exhibits \textbf{multiple competing time scales}:

\subsection{Diffusion time scale}
\begin{equation}
\tau_{\text{diff}} = \frac{L^2}{D_{\text{eff}}}
\end{equation}

\subsection{Electrostatic relaxation time (Maxwell time)}
\begin{equation}
\tau_{\text{Maxwell}} = \frac{\varepsilon}{\sigma} = \frac{\varepsilon RT}{F^2 c D_{\text{eff}}}
\end{equation}

where $\sigma = \frac{F^2 c D_{\text{eff}}}{RT}$ is the ionic conductivity.

\subsection{Physical meaning of permittivity}
\begin{itemize}
\item \textbf{Large $\varepsilon$}: Stronger screening of electric fields $\rightarrow$ slower electrostatic adjustment $\rightarrow$ longer equilibration time
\item \textbf{Small $\varepsilon$}: Weaker screening $\rightarrow$ faster electrostatic response $\rightarrow$ shorter equilibration time  
\end{itemize}

\section{Approximate Solution (Small Perturbation)}

For small voltages, linearization around uniform state gives:
\begin{equation}
\lambda_n \approx \left(\frac{n\pi}{L}\right)^2 \left[D_{\text{eff}} + \frac{DF^2V_{\text{gate}}^2}{R^2T^2} \cdot \text{correction terms}\right]
\end{equation}

Where $D_{\text{eff}}$ is an effective diffusivity combining Li$^+$ and anion transport.

\subsection{Corrected dominant time scale}
The characteristic time for current decay is controlled by the \textbf{slower} of the two processes:
\begin{equation}
\tau_1 \approx \text{max}\left(\frac{L^2}{\pi^2 D_{\text{eff}}}, \frac{\varepsilon RT}{F^2 c D_{\text{eff}}}\right) \times [1 + \text{coupling corrections}]
\end{equation}

\subsection{Regime classification}
\begin{itemize}
\item \textbf{Diffusion-limited}: $\tau_{\text{diff}} > \tau_{\text{Maxwell}}$ $\rightarrow$ transport limited by ion mobility
\item \textbf{Electrostatic-limited}: $\tau_{\text{Maxwell}} > \tau_{\text{diff}}$ $\rightarrow$ transport limited by field relaxation
\end{itemize}

\section{Key Features}

\begin{enumerate}
\item \textbf{Multi-exponential decay} with different time constants $\tau_n = 1/\lambda_n$
\item \textbf{Fastest mode} ($\lambda_1$) dominates long-term behavior  
\item \textbf{Coupling} between concentration and potential evolution
\item \textbf{Current decay} follows the same exponential modes
\item \textbf{Steady state}: Current approaches zero as $t \to \infty$
\end{enumerate}

\section{Parameter Dependencies}

\subsection{Current parameters}
\begin{align}
V_{\text{gate}} &= 0.01 \text{ V} \\
L &= 1 \times 10^{-7} \text{ m} \\
D_{\text{Li}} &= 2.4 \times 10^{-12} \text{ m}^2\text{/s} \\
D_a &= 2.0 \times 10^{-12} \text{ m}^2\text{/s} \\
\frac{F}{RT} &= 38.7 \text{ V}^{-1}
\end{align}

\subsection{Expected concentration variations}
At steady state:
\begin{align}
c_{\text{Li}}(0) &= c_{\text{Li}}^0 \exp(-0.387) = 0.679 \times c_{\text{Li}}^0 \quad \text{(32\% depletion)} \\
c_a(0) &= c_a^0 \exp(+0.387) = 1.472 \times c_a^0 \quad \text{(47\% accumulation)}
\end{align}

\subsection{Numerical time scale calculation}
With current parameters:
\begin{align}
\tau_{\text{diff}} &= \frac{L^2}{D_{\text{eff}}} = \frac{(10^{-8})^2}{2.2 \times 10^{-12}} \approx 4.5 \times 10^{-5} \text{ s} \\
\tau_{\text{Maxwell}} &= \frac{\varepsilon RT}{F^2 c D_{\text{eff}}} = \frac{(30 \times 8.85 \times 10^{-12}) \times 8.314 \times 300}{(96485)^2 \times 110 \times 2.2 \times 10^{-12}} \\
&\approx 2.7 \times 10^{-6} \text{ s}
\end{align}

Since $\tau_{\text{diff}} > \tau_{\text{Maxwell}}$, the system is in the \textbf{diffusion-limited regime}. The characteristic time for current decay is:
\begin{equation}
\tau_1 \approx \tau_{\text{diff}} \approx 4.5 \times 10^{-5} \text{ s} = 0.045 \text{ ms}
\end{equation}

This explains why the current reaches a plateau quickly and requires very small time steps to resolve the transient behavior accurately.

\end{document}